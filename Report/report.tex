\documentclass[UTF-8,a4paper,10pt]{article}

\usepackage{geometry}
\usepackage{graphicx}
\usepackage{newtxmath}
\usepackage{hyperref}
\usepackage{comment}
\usepackage{subfigure}
\usepackage{caption}

\newcommand{\titlemark}{A Report about Deep Learning on MURA Dataset}

\geometry{top=2.54cm,bottom=2.54cm,left=3.17cm,right=3.17cm}
%\hypersetup{hidelinks}
\bibliographystyle{plain}

\def\UrlBreaks{\do\A\do\B\do\C\do\D\do\E\do\F\do\G\do\H\do\I\do\J\do\K\do\L\do\M\do\N\do\O\do\P\do\Q\do\R\do\S\do\T\do\U\do\V\do\W\do\X\do\Y\do\Z\do\[\do\\\do\]\do\^\do\_\do\`\do\a\do\b\do\c\do\d\do\e\do\f\do\g\do\h\do\i\do\j\do\k\do\l\do\m\do\n\do\o\do\p\do\q\do\r\do\s\do\t\do\u\do\v\do\w\do\x\do\y\do\z\do\0\do\1\do\2\do\3\do\4\do\5\do\6\do\7\do\8\do\9\do\.\do\@\do\\\do\/\do\!\do\_\do\|\do\;\do\>\do\]\do\)\do\,\do\?\do\'\do+\do\=\do\#}

\title{\titlemark}
\author{%
\begin{tabular}{c}
Jinchen Xuan, Xingyu Ni, Zhijian Duan 
\end{tabular}%
}

\begin{document}

\maketitle

\section{Introduction}
MURA is a large dataset of musculoskeletal radiographs containing 40,561 images from 14,863 studies\cite{Rajpurkar2017MURA}, where each study is manually labeled by radiologists as either normal or abnormal.

In order to develop an effective and available algorithm, we has trained several networks and done many experiments. Here is our report.

\section{Our Work}

\subsection{Model Training}

For each image input, we resized it to a uniform size. Because input images differ in specification, we have done center crop, random horizontal flip and random rotation. Firstly, we tried to train an 161-layer, an 169-layer and a 201-layer DenseNet\cite{Huang2017Densely} respectively.

We noticed that many images have been labeled to mark ordinal or position, which were very bright in the heat map (see section~2.2). We believed thar they have nothing to do with our problem, so we also tried to remove watermark in the processing.

Based on intuitive guesswork, we thought that training the difficult data first can improve the quality of the model. We selected 8000 images with the lowest loss in a previous model and trained them before training the whole dataset.

Lastly, we binded several models together. The results of all the above models are listed in the table below.

\begin{table}[!ht]
\centering
\begin{tabular}{|c|c|c|c|c|}
\hline
\textbf{ID} & \textbf{Network} & \textbf{Description} & \textbf{Loss Function} & \textbf{Accuracy}\\
\hline
1 & DenseNet-161 & & log loss & \\ 
\hline
2 & DenseNet-169 & & log loss & \\ 
\hline
3 & DenseNet-201 & & log loss & \\ 
\hline
\end{tabular}
\caption{Comparison of different models}
\end{table}

\subsection{Explanatory Analysis}

\subsection{Image Retrieval}

\section{Summary}

\bibliography{report}

\end{document}
